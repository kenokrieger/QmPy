%% Generated by Sphinx.
\def\sphinxdocclass{report}
\documentclass[letterpaper,12,english]{sphinxmanual}
\ifdefined\pdfpxdimen
   \let\sphinxpxdimen\pdfpxdimen\else\newdimen\sphinxpxdimen
\fi \sphinxpxdimen=.75bp\relax

\PassOptionsToPackage{warn}{textcomp}
\usepackage[utf8]{inputenc}
\ifdefined\DeclareUnicodeCharacter
% support both utf8 and utf8x syntaxes
  \ifdefined\DeclareUnicodeCharacterAsOptional
    \def\sphinxDUC#1{\DeclareUnicodeCharacter{"#1}}
  \else
    \let\sphinxDUC\DeclareUnicodeCharacter
  \fi
  \sphinxDUC{00A0}{\nobreakspace}
  \sphinxDUC{2500}{\sphinxunichar{2500}}
  \sphinxDUC{2502}{\sphinxunichar{2502}}
  \sphinxDUC{2514}{\sphinxunichar{2514}}
  \sphinxDUC{251C}{\sphinxunichar{251C}}
  \sphinxDUC{2572}{\textbackslash}
\fi
\usepackage{cmap}
\usepackage[T1]{fontenc}
\usepackage{amsmath,amssymb,amstext}
\usepackage{babel}



\setmainfont{DejaVu Serif}
\setsansfont{DejaVu Sans}
\setmonofont{DejaVu Sans Mono}


\usepackage[Bjornstrup]{fncychap}
\usepackage{sphinx}

\fvset{fontsize=\small}
\usepackage{geometry}


% Include hyperref last.
\usepackage{hyperref}
% Fix anchor placement for figures with captions.
\usepackage{hypcap}% it must be loaded after hyperref.
% Set up styles of URL: it should be placed after hyperref.
\urlstyle{same}

\addto\captionsenglish{\renewcommand{\contentsname}{Contents:}}

\usepackage{sphinxmessages}
\setcounter{tocdepth}{1}


\usepackage[titles]{tocloft}
\cftsetpnumwidth {1.25cm}\cftsetrmarg{1.5cm}
\setlength{\cftchapnumwidth}{0.75cm}
\setlength{\cftsecindent}{\cftchapnumwidth}
\setlength{\cftsecnumwidth}{1.25cm}


\title{QmPy}
\date{Aug 06, 2020}
\release{}
\author{Helmut Wecke, Keno Krieger}
\newcommand{\sphinxlogo}{\vbox{}}
\renewcommand{\releasename}{}
\makeindex
\begin{document}

\pagestyle{empty}
\sphinxmaketitle
\pagestyle{plain}
\sphinxtableofcontents
\pagestyle{normal}
\phantomsection\label{\detokenize{index::doc}}



\chapter{API\sphinxhyphen{}Documentation}
\label{\detokenize{api:module-solvers}}\label{\detokenize{api:api-documentation}}\label{\detokenize{api::doc}}\index{module@\spxentry{module}!solvers@\spxentry{solvers}}\index{solvers@\spxentry{solvers}!module@\spxentry{module}}
Contains numerical solvers for the schroedinger equation
\index{calculate\_expval() (in module solvers)@\spxentry{calculate\_expval()}\spxextra{in module solvers}}

\begin{fulllineitems}
\phantomsection\label{\detokenize{api:solvers.calculate_expval}}\pysiglinewithargsret{\sphinxcode{\sphinxupquote{solvers.}}\sphinxbfcode{\sphinxupquote{calculate\_expval}}}{\emph{\DUrole{n}{xcoordsarray}}, \emph{\DUrole{n}{wfuncsarray}}, \emph{\DUrole{n}{xmin}}, \emph{\DUrole{n}{xmax}}, \emph{\DUrole{n}{npoints}}}{}
Calculates the expected values for the x\sphinxhyphen{}coordinate
\begin{quote}\begin{description}
\item[{Parameters}] \leavevmode\begin{itemize}
\item {} 
\sphinxstyleliteralstrong{\sphinxupquote{xcoordsarray}} (\sphinxstyleliteralemphasis{\sphinxupquote{1darray}}) \textendash{} Array containing the x\sphinxhyphen{}coordinates

\item {} 
\sphinxstyleliteralstrong{\sphinxupquote{wfuncsarray}} (\sphinxstyleliteralemphasis{\sphinxupquote{ndarray}}) \textendash{} Array containing the wave functions that

\item {} 
\sphinxstyleliteralstrong{\sphinxupquote{to the x\sphinxhyphen{}coordinates}} (\sphinxstyleliteralemphasis{\sphinxupquote{correspond}}) \textendash{} 

\item {} 
\sphinxstyleliteralstrong{\sphinxupquote{xmin}} (\sphinxstyleliteralemphasis{\sphinxupquote{float}}) \textendash{} Minimal value of the x\sphinxhyphen{}axis

\item {} 
\sphinxstyleliteralstrong{\sphinxupquote{xmax}} (\sphinxstyleliteralemphasis{\sphinxupquote{float}}) \textendash{} Maximal value of the x\sphinxhyphen{}axis

\item {} 
\sphinxstyleliteralstrong{\sphinxupquote{npoints}} (\sphinxstyleliteralemphasis{\sphinxupquote{int}}) \textendash{} Number of points in the interval {[}xmin, xmax{]}

\end{itemize}

\item[{Returns}] \leavevmode
The expected values of the x\sphinxhyphen{}coordinate

\item[{Return type}] \leavevmode
expval (1darray)

\end{description}\end{quote}

\end{fulllineitems}

\index{calculate\_uncertainity() (in module solvers)@\spxentry{calculate\_uncertainity()}\spxextra{in module solvers}}

\begin{fulllineitems}
\phantomsection\label{\detokenize{api:solvers.calculate_uncertainity}}\pysiglinewithargsret{\sphinxcode{\sphinxupquote{solvers.}}\sphinxbfcode{\sphinxupquote{calculate\_uncertainity}}}{\emph{\DUrole{n}{xcoordsarray}}, \emph{\DUrole{n}{wfuncsarray}}, \emph{\DUrole{n}{xmin}}, \emph{\DUrole{n}{xmax}}, \emph{\DUrole{n}{npoints}}}{}
Calculates the uncertainity (which is the square root of the expected
value of x**2 minus the square of the expected value of x) for
the x\sphinxhyphen{}coordinate
\begin{quote}\begin{description}
\item[{Parameters}] \leavevmode\begin{itemize}
\item {} 
\sphinxstyleliteralstrong{\sphinxupquote{xcoordsarray}} (\sphinxstyleliteralemphasis{\sphinxupquote{1darray}}) \textendash{} Array containing the x\sphinxhyphen{}coordinates

\item {} 
\sphinxstyleliteralstrong{\sphinxupquote{wfuncsarray}} (\sphinxstyleliteralemphasis{\sphinxupquote{ndarray}}) \textendash{} Array containing the wave functions that

\item {} 
\sphinxstyleliteralstrong{\sphinxupquote{to the x\sphinxhyphen{}coordinates}} (\sphinxstyleliteralemphasis{\sphinxupquote{correspond}}) \textendash{} 

\item {} 
\sphinxstyleliteralstrong{\sphinxupquote{xmin}} (\sphinxstyleliteralemphasis{\sphinxupquote{float}}) \textendash{} Minimal value of the x\sphinxhyphen{}axis

\item {} 
\sphinxstyleliteralstrong{\sphinxupquote{xmax}} (\sphinxstyleliteralemphasis{\sphinxupquote{float}}) \textendash{} Maximal value of the x\sphinxhyphen{}axis

\item {} 
\sphinxstyleliteralstrong{\sphinxupquote{npoints}} (\sphinxstyleliteralemphasis{\sphinxupquote{int}}) \textendash{} Number of points in the interval {[}xmin, xmax{]}

\end{itemize}

\item[{Returns}] \leavevmode
The expected values of the x\sphinxhyphen{}coordinate

\item[{Return type}] \leavevmode
uncertainity (1darray)

\end{description}\end{quote}

\end{fulllineitems}

\index{schroedinger() (in module solvers)@\spxentry{schroedinger()}\spxextra{in module solvers}}

\begin{fulllineitems}
\phantomsection\label{\detokenize{api:solvers.schroedinger}}\pysiglinewithargsret{\sphinxcode{\sphinxupquote{solvers.}}\sphinxbfcode{\sphinxupquote{schroedinger}}}{\emph{\DUrole{n}{mass}}, \emph{\DUrole{n}{xcords}}, \emph{\DUrole{n}{potential}}}{}
Solves the 1\sphinxhyphen{}dimensional schroedinger equation for given numerical
values of a potential.
\begin{quote}\begin{description}
\item[{Parameters}] \leavevmode\begin{itemize}
\item {} 
\sphinxstyleliteralstrong{\sphinxupquote{mass}} (\sphinxstyleliteralemphasis{\sphinxupquote{float}}) \textendash{} The mass of the system in atomic units.

\item {} 
\sphinxstyleliteralstrong{\sphinxupquote{xcords}} (\sphinxstyleliteralemphasis{\sphinxupquote{1darray}}) \textendash{} X\sphinxhyphen{}coordinates corresponding to the potential
values.

\item {} 
\sphinxstyleliteralstrong{\sphinxupquote{potential}} (\sphinxstyleliteralemphasis{\sphinxupquote{1darray}}) \textendash{} Numerical values of the potential.

\end{itemize}

\item[{Returns}] \leavevmode

\sphinxcode{\sphinxupquote{energies, wfuncs}}
\begin{itemize}
\item {} 
\sphinxstylestrong{energies} (\sphinxstyleemphasis{1darray}) \sphinxhyphen{} The energy levels of each wavefunctions.
The entries correspond to the rows in wfuncs.

\item {} 
\sphinxstylestrong{wfuncs} (\sphinxstyleemphasis{ndarray}) \sphinxhyphen{} Array where each row contains the numerical
value of a computed wavefunction. Each column corresponds to one
x\sphinxhyphen{}coordinate of the input array.

\end{itemize}


\item[{Return type}] \leavevmode
touple

\end{description}\end{quote}

\end{fulllineitems}



\chapter{Indices and tables}
\label{\detokenize{index:indices-and-tables}}\begin{itemize}
\item {} 
\DUrole{xref,std,std-ref}{genindex}

\item {} 
\DUrole{xref,std,std-ref}{modindex}

\item {} 
\DUrole{xref,std,std-ref}{search}

\end{itemize}


\renewcommand{\indexname}{Python Module Index}
\begin{sphinxtheindex}
\let\bigletter\sphinxstyleindexlettergroup
\bigletter{s}
\item\relax\sphinxstyleindexentry{solvers}\sphinxstyleindexpageref{api:\detokenize{module-solvers}}
\end{sphinxtheindex}

\renewcommand{\indexname}{Index}
\footnotesize\raggedright\printindex
\end{document}